\documentclass[12pt,a4paper]{article}


\usepackage{fontspec}
\usepackage[georgian, english]{babel} 


\setmainfont{FreeSerif}


\usepackage{geometry}
\geometry{margin=1in}
\usepackage{booktabs}
\usepackage{hyperref}
\usepackage{titlesec}
\usepackage{setspace} 
\usepackage{enumitem}
\usepackage{graphicx}


\onehalfspacing


\titleformat{\section}{\normalfont\Large\bfseries}{\thesection}{1em}{}[{\titlerule[0.8pt]}]


\title{\textbf ვებ 4.0: სიმბიოზური ინტერნეტის ეპოქა}
\author{რატი კოჭუაშვილი / Rati kotchuashvili \\ \small კომპიუტერული მეცნიერებების ფაკულტეტი}


\begin{document}

\maketitle

\begin{abstract}
\begin{otherlanguage}{georgian}
მოცემული ნაშრომი განიხილავს ინტერნეტის ევოლუციის მეოთხე ფაზას — Web 4.0-ს. ნაშრომში გაანალიზებულია გადასვლა სტატიკური და სოციალური ქსელებიდან ინტელექტუალურ, ავტონომიურ სისტემებზე. განხილულია ხელოვნური ინტელექტის (AI), ნივთების ინტერნეტისა (IoT) და ტვინ-კომპიუტერის ინტერფეისების (BCI) როლი ადამიანისა და მანქანის სიმბიოზში.
\end{otherlanguage}
\end{abstract}

\tableofcontents
\newpage

\section{Introduction / შესავალი}
\begin{otherlanguage}{georgian}


ინტერნეტის განვითარების ისტორია არის გზა მარტივი კომუნიკაციიდან რთულ კოგნიტურ სისტემებამდე. Web 1.0 იყო ინფორმაციული, Web 2.0 — ინტერაქტიული, Web 3.0 — დეცენტრალიზებული, ხოლო Web 4.0 წარმოადგენს „აქტიურ“ ვებს, სადაც ტექნოლოგია  ადამიანის ინტელექტის აგრძელებს.
\end{otherlanguage}



\section{Evolutionary Framework / ევოლუციური ჩარჩო}
\begin{otherlanguage}{georgian}
\begin{figure}[h]
    \centering
    \includegraphics[width=0.8\textwidth]{Images/1.1.jpeg} 
    \caption{\begin{otherlanguage}{georgian}ვებ 4.0-ის ევოლუციის სქემა\end{otherlanguage}}
\end{figure}
ვების განვითარება შეგვიძლია დავყოთ შემდეგ ეტაპებად:
\end{otherlanguage}

\begin{itemize}
    \item \textbf{Web 1.0 (The Information Web):} \begin{otherlanguage}{georgian}სტატიკური HTML გვერდები, სადაც მომხმარებელი მხოლოდ პასიური მკითხველი იყო.\end{otherlanguage}
    \item \textbf{Web 2.0 (The Social Web):} \begin{otherlanguage}{georgian}მომხმარებლის მიერ გენერირებული კონტენტი (Social Media) და ღრუბლოვანი სერვისები.\end{otherlanguage}
    \item \textbf{Web 3.0 (The Semantic Web):} \begin{otherlanguage}{georgian}მონაცემთა დეცენტრალიზაცია, ბლოკჩეინი და მანქანური კითხვადობა.\end{otherlanguage}
    \item \textbf{Web 4.0 (The Symbiotic Web):} \begin{otherlanguage}{georgian}ადამიანისა და კომპიუტერის რეალურ დროში ურთიერთქმედება.\end{otherlanguage}
\end{itemize}

\section{Core Pillars of Web 4.0 / ძირითადი სვეტები}

\begin{figure}[h]
    \centering
    \includegraphics[width=0.8\textwidth]{Images/1.2.jpeg} 
    \caption{\begin{otherlanguage}{georgian}ვებ 4.0 სქემა\end{otherlanguage}}
\end{figure}

\subsection{Artificial General Intelligence (AGI)}
\begin{otherlanguage}{georgian}
Web 4.0-ის გული არის ხელოვნური ინტელექტი. განსხვავებით წინა თაობებისგან, აქ AI არამხოლოდ ამუშავებს მონაცემებს, არამედ იღებს გადაწყვეტილებებს და ასრულებს ავტონომიურ მოქმედებებს მომხმარებლის ინტერესებიდან გამომდინარე.
\end{otherlanguage}

\subsection{Brain-Computer Interface (BCI)}
\begin{otherlanguage}{georgian}
ეს არის ყველაზე ინოვაციური კომპონენტი. ტვინ-კომპიუტერის ინტერფეისი (მაგალითად, Neuralink) საშუალებას გვაძლევს ინტერნეტთან კავშირი გვქონდეს ფიზიკური მოწყობილობების (კლავიატურა, მაუსი) გარეშე, უშუალოდ ნეირონული იმპულსების მეშვეობით.
\end{otherlanguage}



\section{Comparative Analysis / შედარებითი ანალიზი}

\begin{table}[h]
\centering
\begin{tabular}{@{}lll@{}}
\toprule
\textbf{Feature} & \textbf{Web 3.0} & \textbf{Web 4.0} \\ \midrule
\begin{otherlanguage}{georgian}მთავარი მიზანი\end{otherlanguage} & Decentralization & Intelligence/Symbiosis \\
\begin{otherlanguage}{georgian}ტექნოლოგია\end{otherlanguage} & Blockchain & Autonomous AI / BCI \\
\begin{otherlanguage}{georgian}მონაცემები\end{otherlanguage} & Semantic Metadata & Neural Data / Real-time IoT \\
\begin{otherlanguage}{georgian}ინტერფეისი\end{otherlanguage} & Crypto Wallets & Neural/Voice/Gesture \\ \bottomrule
\end{tabular}
\caption{Web 3.0 vs Web 4.0 Comparison}
\end{table}

\section{Ethical Challenges / ეთიკური გამოწვევები}
\begin{otherlanguage}{georgian}
Web 4.0-ის დანერგვა ბადებს კრიტიკულ კითხვებს:
\end{otherlanguage}
\begin{enumerate}
    \item \textbf{Privacy (კონფიდენციალურობა):} \begin{otherlanguage}{georgian}როდესაც სისტემას აქვს წვდომა ადამიანის ნეირონულ მონაცემებზე, პირადი სივრცე საფრთხის ქვეშ დგება.\end{otherlanguage}
    \item \textbf{Autonomy (ავტონომიურობა):} \begin{otherlanguage}{georgian}რამდენად უსაფრთხოა გადაწყვეტილების მიღების სრული დელეგირება ალგორითმებზე?\end{otherlanguage}
    \item \textbf{Cybersecurity:} \begin{otherlanguage}{georgian}ნეირო-ჰაკინგი ხდება რეალური საფრთხე.\end{otherlanguage}
\end{enumerate}

\section{Conclusion / დასკვნა}
\begin{otherlanguage}{georgian}
Web 4.0 წარმოადგენს ტექნოლოგიურ რევოლუციას, რომელიც წაშლის ზღვარს ციფრულ და ფიზიკურ სამყაროს შორის. იგი გვთავაზობს უსაზღვრო შესაძლებლობებს მედიცინაში, განათლებასა და ყოველდღიურ ცხოვრებაში, თუმცა მოითხოვს მკაცრ ეთიკურ რეგულაციებს.
\end{otherlanguage}

\begin{thebibliography}{9}
\selectlanguage{english}
\bibitem{schwab} Schwab, K. (2016). \textit{The Fourth Industrial Revolution}. World Economic Forum.
\bibitem{web4} Aghaei, S. (2012). Evolution of the World Wide Web: From Web 1.0 to Web 4.0.
\end{thebibliography}

\end{document}